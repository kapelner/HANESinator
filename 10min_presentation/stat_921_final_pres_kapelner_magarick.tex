\documentclass[presentation]{beamer}

\include{preamble}

%presentation preamble
\usetheme{progressbar}
\usecolortheme{progressbar} 
\usefonttheme{progressbar} 
\useoutertheme{progressbar}
\useinnertheme{progressbar}
%\progressbaroptions{headline=sections, frametitle=normal}

\title[Obs. Study on Sleep]{Analysis of NHANES Sleep Data}
\author[KM]{Adam Kapelner, Josh Magarick \\ {\tiny (Special Thanks to Dylan Small)}}
\institute[Wharton, Statistics]{Department of Statistics \\ The Wharton School, University of Pennsylvania}
\date[11/29/11]{}

\begin{document}

\frame{\titlepage}

%http://www.cdc.gov/nchs/nhanes/nhanes2007-2008/nhanes07_08.htm

\section{NHANES}

\begin{frame}
	\frametitle{NHANES}

Since the National Health Survey Act of 1956, \pause the US thought it was important to get readings on the population's health status. In 1971, this was coined... \pause 

\vspace{0.5cm}

\begin{quotation}
``The National Health and Nutrition Examination Survey (NHANES)... a program of studies designed to assess the health and nutritional status of adults and children in the United States.''
\end{quotation}

\begin{columns}
		\begin{column}{.2\textwidth}
\begin{figure}[htp]
\centering
\includegraphics[width=0.75in]{nhanes_logo.jpg}
%\caption{Point estimation problem figure}
%\label{fig:point_estimation}
\end{figure}
\FloatBarrier
		\end{column}
		\begin{column}{.8\textwidth}
\begin{figure}[htp]
\centering
\includegraphics[width=2.5in]{dohhs_logo.jpg}
%\caption{Point estimation problem figure}
%\label{fig:point_estimation}
\end{figure}
\FloatBarrier
		\end{column}
	\end{columns}

\vspace{0.5cm}

\pause

\footnotesize
Nationally representative sample of about 10,000 people every year by using a stratified sample of households from a sample of geographical units (deliberate oversampling of the elderly and minorities).
\normalsize

\end{frame}

\begin{frame}
	\frametitle{The data (Part I)}

They try to measure various indicators:

\begin{itemize}
\item demographic
\item socioeconomic
\item dietary
\item health-related
\end{itemize}

\pause

The first three indicators are queried via surveys (both in-person interviews and computer-aided). \pause The health related data collection is a combination of surveying and a sophisticated: \pause

\begin{itemize}
\item in-depth physical examination \pause
(e.g. audiometry, \pause all the physical measurements, \pause X-ray absorption, \pause retinal imaging battery, \pause dental decay, etc.)
\end{itemize}

\end{frame}

\begin{frame}
	\frametitle{The data (Part II)}

\begin{itemize}
\item in-depth laboratory tests \pause
(e.g. lead and heavy metals in blood, \pause pesticides in urine, \pause testing for common STDs, \pause thyroid profile, \pause  glucose and insulin) \pause
\end{itemize}

How did they collect this data? \pause Via ``mobile equipment centers'':

\begin{figure}[htp]
\centering
\includegraphics[width=4.5in]{mec.jpg}
%\caption{Point estimation problem figure}
%\label{fig:point_estimation}
\end{figure}
\FloatBarrier

\end{frame}

\begin{frame}
	\frametitle{Why is this useful?}

Even via univariate analyses a la Stat 101, the NHANES data has been instrumental in: \pause

\begin{itemize}
\item Constructing growth charts
\item National mandate to add folate and Iron to cereal
\item Measuring lead poisoning
\item Nationwide efforts to reduce cholesterol
\end{itemize}

\pause

And, using basic regression a la Stat 102, you can detect ``associations'' and actually publish papers. \pause And, using the methods of this class, hopefully we can find causal relationships...

\end{frame}

\section{Sleep}

\begin{frame}
	\frametitle{What are we interested in?}

\pause

\begin{center}
What \textit{causes} \Huge{ Sleep}
\end{center}

\begin{figure}[htp]
\centering
\includegraphics[width=3in]{sleep.jpg}
%\caption{Point estimation problem figure}
%\label{fig:point_estimation}
\end{figure}
\FloatBarrier

Why? \pause Because it's pretty important... and...

\end{frame}

\begin{frame}
	\frametitle{Previous Studies using NHANES and Sleep}

We searched google scholar for ``\texttt{sleep}'' AND ``\texttt{NHANES}'' and of the first 100 results, we found 13 papers... \pause

\begin{itemize}
\item 10 / 13 use sleep as covariate to predict something else \pause
\item NONE of the papers use any methods from this class 
\end{itemize}

$\Rightarrow$ \pause We are interested in proving certain lifestyles \textit{cause} a good (or bad) night's sleep.

\end{frame}

\begin{frame}
	\frametitle{Our Dataset}

\footnotesize
We used the 2007/8 data $n = 9,762$ \pause (chosen because of some covariates which later turned out to be not important, so the choice was arbitrary).  \pause We threw out nutritional data, so $p = 2,355$.  \pause Our Response variable was chosen to be ``\texttt{SLD010H}''. 


\vspace{0.2cm}

\begin{quotation}
How much sleep do you usually get at night on weekdays or workdays? ENTER HOURS. 
\end{quotation}

$n_0 = 6,498 \approx \frac{2}{3}n$ answered this question. \pause What about other covariates?
\normalsize


\begin{figure}[htp]
\centering
\includegraphics[width=2.75in]{cov_by_response.jpg}
%\caption{Point estimation problem figure}
%\label{fig:point_estimation}
\end{figure}
\FloatBarrier

\pause

\end{frame}

\section{The Analysis}

\begin{frame}
	\frametitle{Preprocessing}

\begin{itemize}
\item Download and label all data
\item Merge all data tables  \pause
\item Pick covariates that are not missing for records that have sleep variables \pause
\item Pick a treatment variable ``\texttt{PAQ635}'' \pause  - Walk or bicycle

\begin{quotation}
... I would like to ask you about the usual way you travel to and from places. For example to work, for shopping, to school. Do you walk or use a bicycle for at least 10 minutes continuously to get to and from places?
\end{quotation}  \pause

\item Set intersect all records available for all covariates
\end{itemize}

\end{frame}

\begin{frame}
	\frametitle{What are our control variables?}

\begin{figure}[htp]
\centering
\includegraphics[width=4in]{covariates.jpg}
\end{figure}
\FloatBarrier

\end{frame}



\end{document}
